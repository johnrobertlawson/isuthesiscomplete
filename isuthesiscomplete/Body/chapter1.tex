% Chapter 1 of the Thesis Template File
\chapter{GENERAL INTRODUCTION}
Edward Lorenz quote about butterflies
The Butterfly Effect has passed into popular culture (Palmer ref). 
Rough linkage of title to weather forecasting

The following thesis comprises four papers. The present chapter continues with a review of background and literature relating to topics within the present thesis. Chapter~\ref{chp:paper1} describes mulitple ensemble configurations that simulate two contrasting bow echoes: one well forecast, another poorly forecast. Chapter~\ref{chp:paper4} describes a modification to the Structure Amplitude Location (Wernli ref) method, where composite reflectivity is used (instead of accumulated precipitation) to gauge the skill of the North American Model summertime forecasts in the central United States. This method is then used in Chapter~\ref{chp:paper2} to consider whether decreased horizontal grid spacing (\dx) increases the spread and skill of a bow-echo ensemble forecast. Analysis of the relationship between mesoscale and synoptic-scale predictability of convective systems is shown in Chapter~\ref{paper3}. General conclusions are then presented in Chapter~\ref{chp:conc}.

[look at PhD Prelim papers for more content]

\section{Mesoscale convective systems}
The mesoscale, often defined in meteorology as the scale between 2 and 2000\,km, is the arena in which thunderstorm cells and complexes form.

Single to multicell
Mesoscale convective systems (MCSs) are thunderstorm complexes more than 100\,km long in one dimension.

\section{Bow echoes}
Prog vs serial
Hazards
Appearance on radar (show examples)
Lifecycle and lifetimes
Mechanisms (RIJ, cold pool, microbursts, hail)
Review of previous observational studies
Poorly forecast in SG14
Review of other papers (Wandishin, James, Clark) for BE/MCS forecast skill, diurnal

\section{Predictability horizons}
Length scale to time scale
Chaos - Lorenz etc - brief mention of fractals, clouds and turbulence
Tie last two together for `time horizon'
Butterflies - name of thesis! Predictability of bow echoes
Practical predictability?
Define predictability, vs skill, etc
IC error vs model error
Systematic vs random error
Durran papers, butterfly papers x2
Zhang papers, predictability on mesoscales

\section{Numerical simulation}
Numerical simulation 
How this relates to predictability with extra error introduced
Grid spacing
Spread and skill relationship
Methods in global/regional models

Ensembles - rationale and info from Leutbecher/Palmer paper
Ensemble types (setting up paper 1)
Mixed IC methods
Mixed param methods - what are params, model error, stochastic?
SKEB scheme, findings, applicability to convective scales (SCB)

How to score forecasts: spread (std, DTE), skill 
Issue of using mean, ways around it (PMM, closest-to-mean, object-based)
(Note we use closest-to-median for paper 2 and object-based SAL in paper 2/4)
Previous numerical simulations of bow echoes
Verification vs obs (radar, ASOS etc)
Error in obs ~ model uncertainty

